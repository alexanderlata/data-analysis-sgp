\documentclass[12pt]{article}

\usepackage[utf8]{inputenc}
\usepackage[T2A]{fontenc}
\usepackage[english, russian]{babel}

\usepackage{amsmath, amsfonts, amsthm, amssymb, amsopn, amscd}
\usepackage{enumerate}
\usepackage[mathscr]{eucal}

\usepackage{hyperref}
\hypersetup{unicode=true,final=true,colorlinks=true}

\theoremstyle{remark}
%\newtheorem{exercise}{Упражнение}
\newtheorem{exercise}{}[section]
\renewcommand{\theexercise}{\textbf{\#\arabic{exercise}}}

% 
%   Вероятностные определения
%
\DeclareMathOperator{\cov}{cov}
\DeclareMathOperator{\corr}{corr}
\DeclareMathOperator*{\plim}{plim}
\DeclareMathOperator{\Var}{Var}
\DeclareMathOperator{\VVar}{V}
\newcommand{\StdDev}{s.d.}

%
%  Эконометрические
%
\DeclareMathOperator{\const}{const}
\DeclareMathOperator{\error}{error}
\DeclareMathOperator{\StdError}{s.e.}
\DeclareMathOperator{\HCStdError}{HC-s.e.}
\DeclareMathOperator{\SER}{SER}
\DeclareMathOperator{\DW}{DW}
\DeclareMathOperator{\probit}{probit}
\DeclareMathOperator{\logit}{logit}
\DeclareMathOperator{\gompit}{gompit}
\DeclareMathOperator{\loglog}{loglog}
\DeclareMathOperator{\WhiteNoise}{{WN}}
\DeclareMathOperator{\DurbinWatson}{DW}
\DeclareMathOperator{\VAR}{VAR}
\DeclareMathOperator{\ARMA}{ARMA}
\DeclareMathOperator{\ARIMA}{ARIMA}

%
%  Линейная алгебра
%
\DeclareMathOperator{\rank}{rank}
\DeclareMathOperator{\dimension}{dim}
\DeclareMathOperator{\tr}{tr}
\newcommand{\LinearSpace}{{\mathfrak L}}
\newcommand{\spaceX}{{\mathbb X}}
\newcommand{\spaceY}{{\mathbb Y}}



%
%   Числовые
%
\newcommand{\Complex}{{\mathbb C}}
\newcommand{\N}{\mathbb N}
\newcommand{\Z}{{\mathbb Z}}
\newcommand{\Q}{{\mathbb Q}}
\newcommand{\R}{{\mathbb R}}
\newcommand{\semiaxes}{{\mathbb R_+}}

%
%  Вероятностные
%
\newcommand{\iid}{{i.i.d.}}
\newcommand{\Exp}{{\mathsf E}}
\newcommand{\Gauss}{{\mathscr N}}
\newcommand{\Likelihood}{{\mathcal L}}
\newcommand{\StError}{{s.e.}}
\newcommand{\ConfInterval}{{\mathcal I}}

%
%   Вектора
%
\newcommand{\vconst}{{\mathbf const}}
\newcommand{\vectx}{{\bm x}}
\newcommand{\vecty}{{\bm y}}
\newcommand{\vectz}{{\bm z}}
\newcommand{\vecte}{{\bm e}}
\newcommand{\vectw}{{\bm w}}
\newcommand{\vecth}{{\bm h}}
\newcommand{\vectr}{{\bm r}}
\newcommand{\vectq}{{\bm q}}
\newcommand{\vectf}{{\bm f}}%{\boldsymbol{f}}
\newcommand{\vectu}{{\bm u}}
\newcommand{\vectv}{{\bm v}}
\newcommand{\vectalpha}{{\bm{\alpha}}}
\newcommand{\vectbeta}{{\bm{\beta}}}
\newcommand{\vectgamma}{{\bm{\gamma}}}
\newcommand{\vectdelta}{{\bm{\delta}}}
\newcommand{\vectomega}{{\bm{\omega}}}
\newcommand{\vecttheta}{{\bm{\theta}}}
\newcommand{\vecteta}{{\bm{\eta}}}
\newcommand{\vectpi}{{\bm{\pi}}}
\newcommand{\vectmu}{{\bm{\mu}}}
\newcommand{\vectxi}{{\bm{\xi}}}
\newcommand{\vectX}{{\bm X}}
\newcommand{\vectY}{{\bm Y}}
\newcommand{\vectZ}{{\bm Z}}
\newcommand{\vectones}{{ 1}}

% 
%  Матрицы
%
\newcommand{\Id}{I}
\newcommand{\matrixX}{{\bm X}}
\newcommand{\matrixY}{{\bm Y}}
\newcommand{\matrixU}{{\bm U}}
\newcommand{\matrixV}{{\bm V}}
\newcommand{\matrixR}{{\bm R}}
\newcommand{\matrixZ}{{\bm Z}}
\newcommand{\matrixA}{{\bm A}}
\newcommand{\matrixB}{{\bm B}}
\newcommand{\matrixQ}{{\bm Q}}
\newcommand{\matrixH}{{\bm H}}
\newcommand{\matrixXX}{X}
\newcommand{\matrixGamma}{{\bm{\Gamma}}}
\newcommand{\matrixPi}{{\bm{\Pi}}}

%
% Теоремы, Примеры etc
%
\newtheorem*{teorema}{Теорема}
\newtheorem*{importante}{Важно!}
\newtheorem*{ejemplo}{Пример}
\newtheorem*{definicion}{Определение}

\theoremstyle{remark}
\newtheorem*{remark}{Замечание}


% \theoremstyle{plain}
% \newtheorem*{trm}{Теорема}
% \newtheorem*{mprtnt}{Важно!}
% \newtheorem*{xmpl}{Пример}
% \newtheorem*{dfntn}{Определение}

% \theoremstyle{remark}
% \newtheorem*{rmrk}{Замечание}

\title{Листок 02. Прогнозирование}

\author{Н.В. Артамонов}

\begin{document}

\maketitle

\tableofcontents

\section{Линейная регрессия}

\begin{exercise}
Для набора данных \texttt{sleep75} рассмотрим линейную регрессию 
\begin{center}
	\textbf{sleep на totwrk, age, south, male}.
\end{center}
\begin{enumerate}
	\item Подгоните модель и выведите коэффициенты подогнанной модели
	\item Рассмотрим трёх людей с характеристиками
	\begin{center}
		\begin{tabular}{|l||l|l|l|l|}\hline
			index & totwrk & age & south & male \\ \hline\hline
			0 & 2160 & 32 & 1 & 0 \\
			1 & 1720 & 24 & 0 & 1 \\
			2 & 2390 & 44 & 0 & 1 \\ \hline
		\end{tabular}
	\end{center}
	вычислите прогноз \textbf{sleep} 
	\item На обучающей выборке вычислите метрики подгонки: \(R^2\), 
	MSE, MAE, MAPE, RMSE
\end{enumerate}
\end{exercise}

\begin{exercise}
Для набора данных \texttt{sleep75} рассмотрим линейную регрессию 
\begin{center}
	\textbf{sleep на totwrk, age, south, male, smsa, yngkid, marr}.
\end{center}
\begin{enumerate}
	\item Подгоните модель и выведите коэффициенты подогнанной модели
	\item Рассмотрим трёх людей с характеристиками
	\begin{center}
		\begin{tabular}{|l||l|l|l|l|l|l|l|}\hline
			index & totwrk & age & south & male & smsa & yngkid & marr \\ \hline\hline
			0 & 2150 & 37 & 0 & 1 & 1 & 0 & 1 \\
			1 & 1950 & 28 & 1 & 1 & 0 & 1 & 0 \\
			2 & 2240 & 26 & 0 & 0 & 1 & 0 & 0 \\ \hline
		\end{tabular}
	\end{center}
	вычислите прогноз \textbf{sleep} 
	\item На обучающей выборке вычислите метрики подгонки: \(R^2\), 
	MSE, MAE, MAPE, RMSE
\end{enumerate}
\end{exercise}

\begin{exercise}
Для набора данных \texttt{wage2} рассмотрим линейную регрессию 
\begin{center}
	\textbf{wage на age, IQ, south, married, urban}.
\end{center}
\begin{enumerate}
	\item Подгоните модель и выведите коэффициенты подогнанной модели
	\item Рассмотрим трёх людей с характеристиками
	\begin{center}
		\begin{tabular}{|l||l|l|l|l|l|}\hline
			index & age & IQ & south & married & urban \\ \hline\hline
			0 & 36 & 105 & 1 & 1 & 1 \\
			1 & 29 & 123 & 0 & 1 & 0 \\
			2 & 25 & 112 & 1 & 0 & 1 \\ \hline
		\end{tabular}
	\end{center}
	вычислите прогноз \textbf{wage} 
	\item На обучающей выборке вычислите метрики подгонки: \(R^2\), 
	MSE, MAE, MAPE, RMSE
\end{enumerate}
\end{exercise}

\begin{exercise}
Для набора данных \texttt{wage2} рассмотрим линейную регрессию 
\begin{center}
	\textbf{log(wage) на age, IQ, south, married, urban}.
\end{center}
\begin{enumerate}
	\item Подгоните модель и выведите коэффициенты подогнанной модели
	\item Рассмотрим трёх людей с характеристиками
	\begin{center}
		\begin{tabular}{|l||l|l|l|l|l|}\hline
			index & age & IQ & south & married & urban \\ \hline\hline
			0 & 36 & 105 & 1 & 1 & 1 \\
			1 & 29 & 123 & 0 & 1 & 0 \\
			2 & 25 & 112 & 1 & 0 & 1 \\ \hline
		\end{tabular}
	\end{center}
	вычислите прогноз \textbf{wage} 
	\item На обучающей выборке вычислите метрики подгонки: \(R^2\), 
	MSE, MAE, MAPE, RMSE
\end{enumerate}
\end{exercise}

\begin{exercise}
Для набора данных \texttt{wage1} рассмотрим линейную регрессию 
\begin{center}
	\textbf{wage на exper, female, married, smsa}.
\end{center}
\begin{enumerate}
	\item Подгоните модель и выведите коэффициенты подогнанной модели
	\item Рассмотрим трёх людей с характеристиками
	\begin{center}
		\begin{tabular}{|l||l|l|l|l|}\hline
			index & exper & female & married & smsa \\ \hline\hline
			0 & 5 & 1 & 1 & 1  \\
			1 & 26 & 0 & 0 & 1 \\
			2 & 38 & 1 & 1 & 0 \\ \hline
		\end{tabular}
	\end{center}
	вычислите прогноз \textbf{wage}
	\item На обучающей выборке вычислите метрики подгонки: \(R^2\), 
	MSE, MAE, MAPE, RMSE
\end{enumerate}
\end{exercise}

\begin{exercise}
Для набора данных \texttt{wage1} рассмотрим линейную регрессию 
\begin{center}
	\textbf{log(wage) на exper, female, married, smsa}.
\end{center}
\begin{enumerate}
	\item Подгоните модель и выведите коэффициенты подогнанной модели
	\item Рассмотрим трёх людей с характеристиками
	\begin{center}
		\begin{tabular}{|l||l|l|l|l|}\hline
			index & exper & female & married & smsa \\ \hline\hline
			0 & 5 & 1 & 1 & 1  \\
			1 & 26 & 0 & 0 & 1 \\
			2 & 38 & 1 & 1 & 0 \\ \hline
		\end{tabular}
	\end{center}
	вычислите прогноз \textbf{wage}
	\item На обучающей выборке вычислите метрики подгонки: \(R^2\), 
	MSE, MAE, MAPE, RMSE
\end{enumerate}
\end{exercise}

\begin{exercise}
Для набора данных \texttt{Labour} рассмотрим линейную регрессию 
\begin{center}
	\textbf{output на capital, labour}.
\end{center}
\begin{enumerate}
	\item Подгоните модель и выведите коэффициенты подогнанной модели
	\item Рассмотрим три фирмы с характеристиками
	\begin{center}
		\begin{tabular}{|l||l||l|l|}\hline
			index & capital & labour \\ \hline\hline
			0 & 2.970 & 85 \\
			1 & 10.450 & 60  \\
			2 & 3.850 & 105 \\ \hline
		\end{tabular}
	\end{center}
	вычислите прогноз \textbf{output}
	\item На обучающей выборке вычислите метрики подгонки: \(R^2\), 
	MSE, MAE, MAPE, RMSE
\end{enumerate}
\end{exercise}

\begin{exercise}
Для набора данных \texttt{Labour} рассмотрим линейную регрессию 
\begin{center}
	\textbf{log(output) на log(capital), log(labour)}.
\end{center}
\begin{enumerate}
	\item Подгоните модель и выведите коэффициенты подогнанной модели
	\item Рассмотрим три фирмы с характеристиками
	\begin{center}
		\begin{tabular}{|l||l|l|}\hline
			index & capital & labour \\ \hline\hline
			0 & 2.970 & 85 \\
			1 & 10.450 & 60  \\
			2 & 3.850 & 105 \\ \hline
		\end{tabular}
	\end{center}
	вычислите прогноз \textbf{output}
	\item На обучающей выборке вычислите метрики подгонки: \(R^2\), 
	MSE, MAE, MAPE, RMSE
\end{enumerate}
\end{exercise}

\begin{exercise}
Для набора данных \texttt{Labour} рассмотрим линейную регрессию 
\begin{center}
	\textbf{output на capital, labour, wage}.
\end{center}
\begin{enumerate}
	\item Подгоните модель и выведите коэффициенты подогнанной модели
	\item Рассмотрим три фирмы с характеристиками
	\begin{center}
		\begin{tabular}{|l||l|l|l|}\hline
			index & capital & labour & wage \\ \hline\hline
			0 & 2.970 & 85 & 36.98\\
			1 & 10.450 & 60 & 33.82  \\
			2 & 3.850 & 105 & 40.23\\ \hline
		\end{tabular}
	\end{center}
	вычислите прогноз \textbf{output}
	\item На обучающей выборке вычислите метрики подгонки: \(R^2\), 
	MSE, MAE, MAPE, RMSE
\end{enumerate}
\end{exercise}

\begin{exercise}
Для набора данных \texttt{Labour} рассмотрим линейную регрессию 
\begin{center}
	\textbf{log(output) на log(capital), log(labour), log(wage)}.
\end{center}
\begin{enumerate}
	\item Подгоните модель и выведите коэффициенты подогнанной модели
	\item Рассмотрим три фирмы с характеристиками
	\begin{center}
		\begin{tabular}{|l||l|l|l|}\hline
			index & capital & labour & wage \\ \hline\hline
			0 & 2.970 & 85 & 36.98\\
			1 & 10.450 & 60 & 33.82  \\
			2 & 3.850 & 105 & 40.23\\ \hline
		\end{tabular}
	\end{center}
	вычислите прогноз \textbf{output} для каждой
	\item На обучающей выборке вычислите метрики подгонки: \(R^2\), 
	MSE, MAE, MAPE, RMSE
\end{enumerate}
\end{exercise}

\section{k-NN}

\begin{exercise}
Для набора данных \texttt{sleep75} рассмотрим задачу прогнозирования
для переменных
\begin{center}
	\begin{tabular}{|c|c|}\hline
		зависимая/target & объясняющая/предикторы/features \\ \hline
		sleep & totwrk, age, south, male \\ \hline
	\end{tabular}
\end{center}
\begin{enumerate}
	\item подгоните на исходном датасете модель k-NN с параметрами
	\begin{center}
		\begin{tabular}{|l|l|l|}\hline
		№ & \(k\) & веса \\ \hline
		1 & 5 & uniform \\
		2 & 5 & distant \\
		3 & 10 & uniform \\
		4 & 10 & distant \\ \hline
		\end{tabular}
	\end{center}
	\item Рассмотрим трёх людей с характеристиками
	\begin{center}
		\begin{tabular}{|l||l|l|l|l|}\hline
			index & totwrk & age & south & male \\ \hline\hline
			0 & 2160 & 32 & 1 & 0 \\
			1 & 1720 & 24 & 0 & 1 \\
			2 & 2390 & 44 & 0 & 1 \\ \hline
		\end{tabular}
	\end{center}
	вычислите прогноз \textbf{sleep} по каждой модели
\end{enumerate}
\end{exercise}

\begin{exercise}
Для набора данных \texttt{sleep75} рассмотрим задачу прогнозирования
для переменных
\begin{center}
	\begin{tabular}{|c|c|}\hline
		зависимая/target & объясняющая/предикторы/features \\ \hline
		sleep & totwrk, age, south, male, smsa, yngkid, marr \\ \hline
	\end{tabular}
\end{center}
\begin{enumerate}
	\item подгоните на исходном датасете модель k-NN с параметрами
	\begin{center}
		\begin{tabular}{|l|l|l|}\hline
		№ & \(k\) & веса \\ \hline
		1 & 5 & uniform \\
		2 & 5 & distant \\
		3 & 10 & uniform \\
		4 & 10 & distant \\ \hline
		\end{tabular}
	\end{center}
	\item Рассмотрим трёх людей с характеристиками
	\begin{center}
		\begin{tabular}{|l||l|l|l|l|l|l|l|}\hline
			index & totwrk & age & south & male & smsa & yngkid & marr \\ \hline\hline
			0 & 2150 & 37 & 0 & 1 & 1 & 0 & 1 \\
			1 & 1950 & 28 & 1 & 1 & 0 & 1 & 0 \\
			2 & 2240 & 26 & 0 & 0 & 1 & 0 & 0 \\ \hline
		\end{tabular}
	\end{center}
	вычислите прогноз \textbf{sleep} по каждой модели
\end{enumerate}
\end{exercise}

\begin{exercise}
Для набора данных \texttt{wage2} рассмотрим задачу прогнозирования
для переменных
\begin{center}
	\begin{tabular}{|c|c|}\hline
		зависимая/target & объясняющая/предикторы/features \\ \hline
		wage & age, IQ, south, married, urban \\ \hline
	\end{tabular}
\end{center}
\begin{enumerate}
	\item подгоните на исходном датасете модель k-NN с параметрами
	\begin{center}
		\begin{tabular}{|l|l|l|}\hline
		№ & \(k\) & веса \\ \hline
		1 & 5 & uniform \\
		2 & 5 & distant \\
		3 & 10 & uniform \\
		4 & 10 & distant \\ \hline
		\end{tabular}
	\end{center}
	\item Рассмотрим трёх людей с характеристиками
	\begin{center}
		\begin{tabular}{|l||l|l|l|l|l|}\hline
			index & age & IQ & south & married & urban \\ \hline\hline
			0 & 36 & 105 & 1 & 1 & 1 \\
			1 & 29 & 123 & 0 & 1 & 0 \\
			2 & 25 & 112 & 1 & 0 & 1 \\ \hline
		\end{tabular}
	\end{center}
	вычислите прогноз \textbf{wage} по каждой модели
\end{enumerate}
\end{exercise}

\begin{exercise}
Для набора данных \texttt{wage2} рассмотрим задачу прогнозирования
для переменных
\begin{center}
	\begin{tabular}{|c|c|}\hline
		зависимая/target & объясняющая/предикторы/features \\ \hline
		log(wage) & age, IQ, south, married, urban \\ \hline
	\end{tabular}
\end{center}
\begin{enumerate}
	\item подгоните на исходном датасете модель k-NN с параметрами
	\begin{center}
		\begin{tabular}{|l|l|l|}\hline
		№ & \(k\) & веса \\ \hline
		1 & 5 & uniform \\
		2 & 5 & distant \\
		3 & 10 & uniform \\
		4 & 10 & distant \\ \hline
		\end{tabular}
	\end{center}
	\item Рассмотрим трёх людей с характеристиками
	\begin{center}
		\begin{tabular}{|l||l|l|l|l|l|}\hline
			index & age & IQ & south & married & urban \\ \hline\hline
			0 & 36 & 105 & 1 & 1 & 1 \\
			1 & 29 & 123 & 0 & 1 & 0 \\
			2 & 25 & 112 & 1 & 0 & 1 \\ \hline
		\end{tabular}
	\end{center}
	вычислите прогноз \textbf{wage} по каждой модели
\end{enumerate}
\end{exercise}

\begin{exercise}
Для набора данных \texttt{wage1} рассмотрим задачу прогнозирования
для переменных
\begin{center}
	\begin{tabular}{|c|c|}\hline
		зависимая/target & объясняющая/предикторы/features \\ \hline
		wage & exper, female, married, smsa \\ \hline
	\end{tabular}
\end{center}
\begin{enumerate}
	\item подгоните на исходном датасете модель k-NN с параметрами
	\begin{center}
		\begin{tabular}{|l|l|l|}\hline
		№ & \(k\) & веса \\ \hline
		1 & 5 & uniform \\
		2 & 5 & distant \\
		3 & 10 & uniform \\
		4 & 10 & distant \\ \hline
		\end{tabular}
	\end{center}
	\item Рассмотрим трёх людей с характеристиками
	\begin{center}
		\begin{tabular}{|l||l|l|l|l|}\hline
			index & exper & female & married & smsa \\ \hline\hline
			0 & 5 & 1 & 1 & 1  \\
			1 & 26 & 0 & 0 & 1 \\
			2 & 38 & 1 & 1 & 0 \\ \hline
		\end{tabular}
	\end{center}
	вычислите прогноз \textbf{wage} по каждой модели
\end{enumerate}
\end{exercise}

\begin{exercise}
Для набора данных \texttt{wage1} рассмотрим задачу прогнозирования
для переменных
\begin{center}
	\begin{tabular}{|c|c|}\hline
		зависимая/target & объясняющая/предикторы/features \\ \hline
		log(wage) & exper, female, married, smsa \\ \hline
	\end{tabular}
\end{center}
\begin{enumerate}
	\item подгоните на исходном датасете модель k-NN с параметрами
	\begin{center}
		\begin{tabular}{|l|l|l|}\hline
		№ & \(k\) & веса \\ \hline
		1 & 5 & uniform \\
		2 & 5 & distant \\
		3 & 10 & uniform \\
		4 & 10 & distant \\ \hline
		\end{tabular}
	\end{center}
	\item Рассмотрим трёх людей с характеристиками
	\begin{center}
		\begin{tabular}{|l||l|l|l|l|}\hline
			index & exper & female & married & smsa \\ \hline\hline
			0 & 5 & 1 & 1 & 1  \\
			1 & 26 & 0 & 0 & 1 \\
			2 & 38 & 1 & 1 & 0 \\ \hline
		\end{tabular}
	\end{center}
	вычислите прогноз \textbf{wage} по каждой модели
\end{enumerate}
\end{exercise}

\begin{exercise}
Для набора данных \texttt{Labour} рассмотрим задачу прогнозирования
для переменных
\begin{center}
	\begin{tabular}{|c|c|}\hline
		зависимая/target & объясняющая/предикторы/features \\ \hline
		output & capital, labour \\ \hline
	\end{tabular}
\end{center}
\begin{enumerate}
	\item подгоните на исходном датасете модель k-NN с параметрами
	\begin{center}
		\begin{tabular}{|l|l|l|}\hline
		№ & \(k\) & веса \\ \hline
		1 & 5 & uniform \\
		2 & 5 & distant \\
		3 & 10 & uniform \\
		4 & 10 & distant \\ \hline
		\end{tabular}
	\end{center}
	\item Рассмотрим трёх людей с характеристиками
	\begin{center}
		\begin{tabular}{|l||l||l|l|}\hline
			index & capital & labour \\ \hline\hline
			0 & 2.970 & 85 \\
			1 & 10.450 & 60  \\
			2 & 3.850 & 105 \\ \hline
		\end{tabular}
	\end{center}
	вычислите прогноз \textbf{output} по каждой модели
\end{enumerate}
\end{exercise}

\begin{exercise}
Для набора данных \texttt{Labour} рассмотрим задачу прогнозирования
для переменных
\begin{center}
	\begin{tabular}{|c|c|}\hline
		зависимая/target & объясняющая/предикторы/features \\ \hline
		log(output) & log(capital), log(labour) \\ \hline
	\end{tabular}
\end{center}
\begin{enumerate}
	\item подгоните на исходном датасете модель k-NN с параметрами
	\begin{center}
		\begin{tabular}{|l|l|l|}\hline
		№ & \(k\) & веса \\ \hline
		1 & 5 & uniform \\
		2 & 5 & distant \\
		3 & 10 & uniform \\
		4 & 10 & distant \\ \hline
		\end{tabular}
	\end{center}
	\item Рассмотрим трёх людей с характеристиками
	\begin{center}
		\begin{tabular}{|l||l||l|l|}\hline
			index & capital & labour \\ \hline\hline
			0 & 2.970 & 85 \\
			1 & 10.450 & 60  \\
			2 & 3.850 & 105 \\ \hline
		\end{tabular}
	\end{center}
	вычислите прогноз \textbf{output} по каждой модели
\end{enumerate}
\end{exercise}

\begin{exercise}
Для набора данных \texttt{Labour} рассмотрим задачу прогнозирования
для переменных
\begin{center}
	\begin{tabular}{|c|c|}\hline
		зависимая/target & объясняющая/предикторы/features \\ \hline
		output & capital, labour, wage \\ \hline
	\end{tabular}
\end{center}
\begin{enumerate}
	\item подгоните на исходном датасете модель k-NN с параметрами
	\begin{center}
		\begin{tabular}{|l|l|l|}\hline
		№ & \(k\) & веса \\ \hline
		1 & 5 & uniform \\
		2 & 5 & distant \\
		3 & 10 & uniform \\
		4 & 10 & distant \\ \hline
		\end{tabular}
	\end{center}
	\item Рассмотрим трёх людей с характеристиками
	\begin{center}
		\begin{tabular}{|l||l|l|l|}\hline
			index & capital & labour & wage \\ \hline\hline
			0 & 2.970 & 85 & 36.98\\
			1 & 10.450 & 60 & 33.82  \\
			2 & 3.850 & 105 & 40.23\\ \hline
		\end{tabular}
	\end{center}
	вычислите прогноз \textbf{output} по каждой модели
\end{enumerate}
\end{exercise}

\begin{exercise}
Для набора данных \texttt{Labour} рассмотрим задачу прогнозирования
для переменных
\begin{center}
	\begin{tabular}{|c|c|}\hline
		зависимая/target & объясняющая/предикторы/features \\ \hline
		log(output) & log(capital), log(labour), log(wage) \\ \hline
	\end{tabular}
\end{center}
\begin{enumerate}
	\item подгоните на исходном датасете модель k-NN с параметрами
	\begin{center}
		\begin{tabular}{|l|l|l|}\hline
		№ & \(k\) & веса \\ \hline
		1 & 5 & uniform \\
		2 & 5 & distant \\
		3 & 10 & uniform \\
		4 & 10 & distant \\ \hline
		\end{tabular}
	\end{center}
	\item Рассмотрим трёх людей с характеристиками
	\begin{center}
		\begin{tabular}{|l||l|l|l|}\hline
			index & capital & labour & wage \\ \hline\hline
			0 & 2.970 & 85 & 36.98\\
			1 & 10.450 & 60 & 33.82  \\
			2 & 3.850 & 105 & 40.23\\ \hline
		\end{tabular}
	\end{center}
	вычислите прогноз \textbf{output} по каждой модели
\end{enumerate}
\end{exercise}


\section{Валидация моделей}

\begin{exercise}
Набор данных \texttt{sleep75} разбейте на обучающую и тестовую часть
в соотношении 80:20.

Рассмотрим задачу прогнозирования для переменных
\begin{center}
	\begin{tabular}{|c|c|}\hline
		зависимая/target & объясняющая/предикторы/features \\ \hline
		sleep & totwrk, age, south, male \\ \hline
	\end{tabular}
\end{center}
и следующие модели
\begin{center}
	\begin{tabular}{|l|l|}\hline
		№ & Модель \\ \hline
		1 & линейная регрессия\\
		2 & k-NN с \(k=5\), веса 'uniform' \\
		3 & k-NN с \(k=5\), веса 'distant' \\
		4 & k-NN с \(k=10\), веса 'uniform' \\
		5 & k-NN с \(k=10\), веса 'distant' \\ \hline
	\end{tabular}
\end{center}
Проведите валидацию моделей относительно метрик \(R^2\), MSE, MAE,
MAPE. Какая модель предпочтительней?
\end{exercise}

\begin{exercise}
Набор данных \texttt{sleep75} разбейте на обучающую и тестовую часть
в соотношении 80:20.

Рассмотрим задачу прогнозирования для переменных
\begin{center}
	\begin{tabular}{|c|c|}\hline
		зависимая/target & объясняющая/предикторы/features \\ \hline
		sleep & totwrk, age, south, male, smsa, yngkid, marr \\ \hline
	\end{tabular}
\end{center}
и следующие модели
\begin{center}
	\begin{tabular}{|l|l|}\hline
		№ & Модель \\ \hline
		1 & линейная регрессия\\
		2 & k-NN с \(k=5\), веса 'uniform' \\
		3 & k-NN с \(k=5\), веса 'distant' \\
		4 & k-NN с \(k=10\), веса 'uniform' \\
		5 & k-NN с \(k=10\), веса 'distant' \\ \hline
	\end{tabular}
\end{center}
Проведите валидацию моделей относительно метрик \(R^2\), MSE, MAE,
MAPE. Какая модель предпочтительней?
\end{exercise}

\begin{exercise}
Набор данных \texttt{wage2} разбейте на обучающую и тестовую часть
в соотношении 80:20.

Рассмотрим задачу прогнозирования для переменных
\begin{center}
	\begin{tabular}{|c|c|}\hline
		зависимая/target & объясняющая/предикторы/features \\ \hline
		wage & age, IQ, south, married, urban \\ \hline
	\end{tabular}
\end{center}
и следующие модели
\begin{center}
	\begin{tabular}{|l|l|}\hline
		№ & Модель \\ \hline
		1 & линейная регрессия\\
		2 & k-NN с \(k=5\), веса 'uniform' \\
		3 & k-NN с \(k=5\), веса 'distant' \\
		4 & k-NN с \(k=10\), веса 'uniform' \\
		5 & k-NN с \(k=10\), веса 'distant' \\ \hline
	\end{tabular}
\end{center}
Проведите валидацию моделей относительно метрик \(R^2\), MSE, MAE,
MAPE. Какая модель предпочтительней?
\end{exercise}

\begin{exercise}
Набор данных \texttt{wage2} разбейте на обучающую и тестовую часть
в соотношении 80:20.

Рассмотрим задачу прогнозирования для переменных
\begin{center}
	\begin{tabular}{|c|c|}\hline
		зависимая/target & объясняющая/предикторы/features \\ \hline
		log(wage) & age, IQ, south, married, urban \\ \hline
	\end{tabular}
\end{center}
и следующие модели
\begin{center}
	\begin{tabular}{|l|l|}\hline
		№ & Модель \\ \hline
		1 & линейная регрессия\\
		2 & k-NN с \(k=5\), веса 'uniform' \\
		3 & k-NN с \(k=5\), веса 'distant' \\
		4 & k-NN с \(k=10\), веса 'uniform' \\
		5 & k-NN с \(k=10\), веса 'distant' \\ \hline
	\end{tabular}
\end{center}
Проведите валидацию моделей относительно метрик \(R^2\), MSE, MAE,
MAPE. Какая модель предпочтительней?
\end{exercise}

\begin{exercise}
Набор данных \texttt{wage1} разбейте на обучающую и тестовую часть
в соотношении 80:20.

Рассмотрим задачу прогнозирования для переменных
\begin{center}
	\begin{tabular}{|c|c|}\hline
		зависимая/target & объясняющая/предикторы/features \\ \hline
		wage & exper, female, married, smsa \\ \hline
	\end{tabular}
\end{center}
и следующие модели
\begin{center}
	\begin{tabular}{|l|l|}\hline
		№ & Модель \\ \hline
		1 & линейная регрессия\\
		2 & k-NN с \(k=5\), веса 'uniform' \\
		3 & k-NN с \(k=5\), веса 'distant' \\
		4 & k-NN с \(k=10\), веса 'uniform' \\
		5 & k-NN с \(k=10\), веса 'distant' \\ \hline
	\end{tabular}
\end{center}
Проведите валидацию моделей относительно метрик \(R^2\), MSE, MAE,
MAPE. Какая модель предпочтительней?
\end{exercise}

\begin{exercise}
Набор данных \texttt{wage1} разбейте на обучающую и тестовую часть
в соотношении 80:20.

Рассмотрим задачу прогнозирования для переменных
\begin{center}
	\begin{tabular}{|c|c|}\hline
		зависимая/target & объясняющая/предикторы/features \\ \hline
		log(wage) & exper, female, married, smsa \\ \hline
	\end{tabular}
\end{center}
и следующие модели
\begin{center}
	\begin{tabular}{|l|l|}\hline
		№ & Модель \\ \hline
		1 & линейная регрессия\\
		2 & k-NN с \(k=5\), веса 'uniform' \\
		3 & k-NN с \(k=5\), веса 'distant' \\
		4 & k-NN с \(k=10\), веса 'uniform' \\
		5 & k-NN с \(k=10\), веса 'distant' \\ \hline
	\end{tabular}
\end{center}
Проведите валидацию моделей относительно метрик \(R^2\), MSE, MAE,
MAPE. Какая модель предпочтительней?
\end{exercise}

\begin{exercise}
Набор данных \texttt{Labour} разбейте на обучающую и тестовую часть
в соотношении 80:20.

Рассмотрим задачу прогнозирования для переменных
\begin{center}
	\begin{tabular}{|c|c|}\hline
		зависимая/target & объясняющая/предикторы/features \\ \hline
		output & capital, labour, wage \\ \hline
	\end{tabular}
\end{center}
и следующие модели
\begin{center}
	\begin{tabular}{|l|l|}\hline
		№ & Модель \\ \hline
		1 & линейная регрессия\\
		2 & k-NN с \(k=5\), веса 'uniform' \\
		3 & k-NN с \(k=5\), веса 'distant' \\
		4 & k-NN с \(k=10\), веса 'uniform' \\
		5 & k-NN с \(k=10\), веса 'distant' \\ \hline
	\end{tabular}
\end{center}
Проведите валидацию моделей относительно метрик \(R^2\), MSE, MAE,
MAPE. Какая модель предпочтительней?
\end{exercise}

\begin{exercise}
Набор данных \texttt{Labour} разбейте на обучающую и тестовую часть
в соотношении 80:20.

Рассмотрим задачу прогнозирования для переменных
\begin{center}
	\begin{tabular}{|c|c|}\hline
		зависимая/target & объясняющая/предикторы/features \\ \hline
		log(output) & log(capital), log(labour), log(wage) \\ \hline
	\end{tabular}
\end{center}
и следующие модели
\begin{center}
	\begin{tabular}{|l|l|}\hline
		№ & Модель \\ \hline
		1 & линейная регрессия\\
		2 & k-NN с \(k=5\), веса 'uniform' \\
		3 & k-NN с \(k=5\), веса 'distant' \\
		4 & k-NN с \(k=10\), веса 'uniform' \\
		5 & k-NN с \(k=10\), веса 'distant' \\ \hline
	\end{tabular}
\end{center}
Проведите валидацию моделей относительно метрик \(R^2\), MSE, MAE,
MAPE. Какая модель предпочтительней?
\end{exercise}

% \section{Валидация моделей и преобразования переменных}

% \begin{exercise}

% \end{exercise}

\end{document}