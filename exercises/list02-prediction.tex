\documentclass[12pt]{article}

\usepackage[utf8]{inputenc}
\usepackage[T2A]{fontenc}
\usepackage[english, russian]{babel}

\usepackage{amsmath, amsfonts, amsthm, amssymb, amsopn, amscd}
\usepackage{enumerate}
\usepackage[mathscr]{eucal}

\usepackage{hyperref}
\hypersetup{unicode=true,final=true,colorlinks=true}

\theoremstyle{remark}
%\newtheorem{exercise}{Упражнение}
\newtheorem{exercise}{}[section]
\renewcommand{\theexercise}{\textbf{\#\arabic{exercise}}}

\input{data-analysis-defs.tex}

\title{Листок 02. Прогнозирование}

\author{Н.В. Артамонов}

\begin{document}

\maketitle

\tableofcontents

\section{Линейная регрессия}

\begin{exercise}
Для набора данных \texttt{sleep75} рассмотрим линейную регрессию 
\begin{center}
	\textbf{sleep на totwrk, age, south, male, smsa, yngkid, marr}.
\end{center}
\begin{enumerate}
	\item Подгоните модель и выведите коэффициенты подогнанной модели
	\item Рассмотрим трёх людей с характеристиками
	\begin{center}
		\begin{tabular}{|l|l|l|l|l|l|l|}\hline
			totwrk & age & south & male & smsa & yngkid & marr \\ \hline\hline
			2150 & 37 & 0 & 1 & 1 & 0 & 1  \\
			1950 & 28 & 1 & 1 & 0 & 1 & 0 \\  
			2240 & 26 & 0 & 0 & 1 & 0 & 0 \\ \hline
		\end{tabular}
	\end{center}
	\item На обучающей выборке вычислите метрики подгонки: \(R^2\), 
	MSE, MAE, MAPE, RMSE
\end{enumerate}
\end{exercise}



\end{document}