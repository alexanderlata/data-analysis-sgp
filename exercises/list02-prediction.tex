\documentclass[12pt]{article}

\usepackage[utf8]{inputenc}
\usepackage[T2A]{fontenc}
\usepackage[english, russian]{babel}

\usepackage{amsmath, amsfonts, amsthm, amssymb, amsopn, amscd}
\usepackage{enumerate}
\usepackage[mathscr]{eucal}

\usepackage{hyperref}
\hypersetup{unicode=true,final=true,colorlinks=true}

\theoremstyle{remark}
%\newtheorem{exercise}{Упражнение}
\newtheorem{exercise}{}[section]
\renewcommand{\theexercise}{\textbf{\#\arabic{exercise}}}

\input{data-analysis-defs.tex}

\title{Листок 02. Прогнозирование}

\author{Н.В. Артамонов}

\begin{document}

\maketitle

\tableofcontents

\section{Линейная регрессия}

\begin{exercise}
Для набора данных \texttt{sleep75} рассмотрим линейную регрессию 
\begin{center}
	\textbf{sleep на totwrk, age, south, male}.
\end{center}
\begin{enumerate}
	\item Подгоните модель и выведите коэффициенты подогнанной модели
	\item Рассмотрим трёх людей с характеристиками
	\begin{center}
		\begin{tabular}{|l|l|l|l|}\hline
			totwrk & age & south & male \\ \hline\hline
			2160 & 32 & 1 & 0 \\
			1720 & 24 & 0 & 1 \\
			2390 & 44 & 0 & 1 \\ \hline
		\end{tabular}
	\end{center}
	вычислите прогноз \textbf{sleep} 
	\item На обучающей выборке вычислите метрики подгонки: \(R^2\), 
	MSE, MAE, MAPE, RMSE
\end{enumerate}
\end{exercise}

\begin{exercise}
Для набора данных \texttt{sleep75} рассмотрим линейную регрессию 
\begin{center}
	\textbf{sleep на totwrk, age, south, male, smsa, yngkid, marr}.
\end{center}
\begin{enumerate}
	\item Подгоните модель и выведите коэффициенты подогнанной модели
	\item Рассмотрим трёх людей с характеристиками
	\begin{center}
		\begin{tabular}{|l|l|l|l|l|l|l|}\hline
			totwrk & age & south & male & smsa & yngkid & marr \\ \hline\hline
			2150 & 37 & 0 & 1 & 1 & 0 & 1 \\
			1950 & 28 & 1 & 1 & 0 & 1 & 0 \\
			2240 & 26 & 0 & 0 & 1 & 0 & 0 \\ \hline
		\end{tabular}
	\end{center}
	вычислите прогноз \textbf{sleep} 
	\item На обучающей выборке вычислите метрики подгонки: \(R^2\), 
	MSE, MAE, MAPE, RMSE
\end{enumerate}
\end{exercise}

\begin{exercise}
Для набора данных \texttt{wage2} рассмотрим линейную регрессию 
\begin{center}
	\textbf{log(wage) на age, IQ, south, married, urban}.
\end{center}
\begin{enumerate}
	\item Подгоните модель и выведите коэффициенты подогнанной модели
	\item Рассмотрим трёх людей с характеристиками
	\begin{center}
		\begin{tabular}{|l|l|l|l|l|}\hline
			age & IQ & south & married & urban \\ \hline\hline
			36 & 105 & 1 & 1 & 1 \\
			29 & 123 & 0 & 1 & 0 \\
			25 & 112 & 1 & 0 & 1 \\ \hline
		\end{tabular}
	\end{center}
	вычислите прогноз \textbf{wage} 
	\item На обучающей выборке вычислите метрики подгонки: \(R^2\), 
	MSE, MAE, MAPE, RMSE
\end{enumerate}
\end{exercise}

\begin{exercise}
Для набора данных \texttt{wage1} рассмотрим линейную регрессию 
\begin{center}
	\textbf{log(wage) на exper, female, married, smsa}.
\end{center}
\begin{enumerate}
	\item Подгоните модель и выведите коэффициенты подогнанной модели
	\item Рассмотрим трёх людей с характеристиками
	\begin{center}
		\begin{tabular}{|l|l|l|l|}\hline
			exper & female & married & smsa \\ \hline\hline
			5 & 1 & 1 & 1  \\
			26 & 0 & 0 & 1 \\
			38 & 1 & 1 & 0 \\ \hline
		\end{tabular}
	\end{center}
	вычислите прогноз \textbf{wage}
	\item На обучающей выборке вычислите метрики подгонки: \(R^2\), 
	MSE, MAE, MAPE, RMSE
\end{enumerate}
\end{exercise}

\begin{exercise}
Для набора данных \texttt{Labour} рассмотрим линейную регрессию 
\begin{center}
	\textbf{output на capital, labour}.
\end{center}
\begin{enumerate}
	\item Подгоните модель и выведите коэффициенты подогнанной модели
	\item Рассмотрим три фирмы с характеристиками
	\begin{center}
		\begin{tabular}{|l|l|}\hline
			capital & labour \\ \hline\hline
			2.970 & 85 \\
			10.450 & 60  \\
			3.850 & 105 \\ \hline
		\end{tabular}
	\end{center}
	вычислите прогноз \textbf{output}
	\item На обучающей выборке вычислите метрики подгонки: \(R^2\), 
	MSE, MAE, MAPE, RMSE
\end{enumerate}
\end{exercise}

\begin{exercise}
Для набора данных \texttt{Labour} рассмотрим линейную регрессию 
\begin{center}
	\textbf{log(output) на log(capital), log(labour)}.
\end{center}
\begin{enumerate}
	\item Подгоните модель и выведите коэффициенты подогнанной модели
	\item Рассмотрим три фирмы с характеристиками
	\begin{center}
		\begin{tabular}{|l|l|}\hline
			capital & labour \\ \hline\hline
			2.970 & 85 \\
			10.450 & 60  \\
			3.850 & 105 \\ \hline
		\end{tabular}
	\end{center}
	вычислите прогноз \textbf{output}
	\item На обучающей выборке вычислите метрики подгонки: \(R^2\), 
	MSE, MAE, MAPE, RMSE
\end{enumerate}
\end{exercise}

\begin{exercise}
Для набора данных \texttt{Labour} рассмотрим линейную регрессию 
\begin{center}
	\textbf{output на capital, labour, wage}.
\end{center}
\begin{enumerate}
	\item Подгоните модель и выведите коэффициенты подогнанной модели
	\item Рассмотрим три фирмы с характеристиками
	\begin{center}
		\begin{tabular}{|l|l|l|}\hline
			capital & labour & wage \\ \hline\hline
			2.970 & 85 & 36.98\\
			10.450 & 60 & 33.82  \\
			3.850 & 105 & 40.23\\ \hline
		\end{tabular}
	\end{center}
	вычислите прогноз \textbf{output}
	\item На обучающей выборке вычислите метрики подгонки: \(R^2\), 
	MSE, MAE, MAPE, RMSE
\end{enumerate}
\end{exercise}

\begin{exercise}
Для набора данных \texttt{Labour} рассмотрим линейную регрессию 
\begin{center}
	\textbf{log(output) на log(capital), log(labour), log(wage)}.
\end{center}
\begin{enumerate}
	\item Подгоните модель и выведите коэффициенты подогнанной модели
	\item Рассмотрим три фирмы с характеристиками
	\begin{center}
		\begin{tabular}{|l|l|l|}\hline
			capital & labour & wage \\ \hline\hline
			2.970 & 85 & 36.98\\
			10.450 & 60 & 33.82  \\
			3.850 & 105 & 40.23\\ \hline
		\end{tabular}
	\end{center}
	вычислите прогноз \textbf{output} для каждой
	\item На обучающей выборке вычислите метрики подгонки: \(R^2\), 
	MSE, MAE, MAPE, RMSE
\end{enumerate}
\end{exercise}

\section{k-NN}

\begin{exercise}
Для набора данных \texttt{sleep75} рассмотрим задачу прогнозирования
\textbf{sleep} (переменная таргета) и помощь предикторов
\textbf{totwrk, age, south, male}
\begin{enumerate}
	\item подгоните модель k-NN с
	\begin{itemize}
		\item \(k=5\)
		\item \(k=10\)
	\end{itemize}
	на исходном датасете
	\item Рассмотрим трёх людей с характеристиками
	\begin{center}
		\begin{tabular}{|l|l|l|l|}\hline
			totwrk & age & south & male \\ \hline\hline
			2160 & 32 & 1 & 0 \\
			1720 & 24 & 0 & 1 \\
			2390 & 44 & 0 & 1 \\ \hline
		\end{tabular}
	\end{center}
	вычислите прогноз \textbf{sleep} по каждой модели
\end{enumerate}
\end{exercise}

\section{Валидация модели}

\end{document}