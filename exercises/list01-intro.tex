\documentclass[12pt]{article}

\usepackage[utf8]{inputenc}
\usepackage[T2A]{fontenc}
\usepackage[english, russian]{babel}

\usepackage{amsmath, amsfonts, amsthm, amssymb, amsopn, amscd}
\usepackage{enumerate}
\usepackage[mathscr]{eucal}

\usepackage{hyperref}
\hypersetup{unicode=true,final=true,colorlinks=true}

\theoremstyle{remark}
%\newtheorem{exercise}{Упражнение}
\newtheorem{exercise}{}[section]
\renewcommand{\theexercise}{\textbf{\#\arabic{exercise}}}

% 
%   Вероятностные определения
%
\DeclareMathOperator{\cov}{cov}
\DeclareMathOperator{\corr}{corr}
\DeclareMathOperator*{\plim}{plim}
\DeclareMathOperator{\Var}{Var}
\DeclareMathOperator{\VVar}{V}
\newcommand{\StdDev}{s.d.}

%
%  Эконометрические
%
\DeclareMathOperator{\const}{const}
\DeclareMathOperator{\error}{error}
\DeclareMathOperator{\StdError}{s.e.}
\DeclareMathOperator{\HCStdError}{HC-s.e.}
\DeclareMathOperator{\SER}{SER}
\DeclareMathOperator{\DW}{DW}
\DeclareMathOperator{\probit}{probit}
\DeclareMathOperator{\logit}{logit}
\DeclareMathOperator{\gompit}{gompit}
\DeclareMathOperator{\loglog}{loglog}
\DeclareMathOperator{\WhiteNoise}{{WN}}
\DeclareMathOperator{\DurbinWatson}{DW}
\DeclareMathOperator{\VAR}{VAR}
\DeclareMathOperator{\ARMA}{ARMA}
\DeclareMathOperator{\ARIMA}{ARIMA}

%
%  Линейная алгебра
%
\DeclareMathOperator{\rank}{rank}
\DeclareMathOperator{\dimension}{dim}
\DeclareMathOperator{\tr}{tr}
\newcommand{\LinearSpace}{{\mathfrak L}}
\newcommand{\spaceX}{{\mathbb X}}
\newcommand{\spaceY}{{\mathbb Y}}



%
%   Числовые
%
\newcommand{\Complex}{{\mathbb C}}
\newcommand{\N}{\mathbb N}
\newcommand{\Z}{{\mathbb Z}}
\newcommand{\Q}{{\mathbb Q}}
\newcommand{\R}{{\mathbb R}}
\newcommand{\semiaxes}{{\mathbb R_+}}

%
%  Вероятностные
%
\newcommand{\iid}{{i.i.d.}}
\newcommand{\Exp}{{\mathsf E}}
\newcommand{\Gauss}{{\mathscr N}}
\newcommand{\Likelihood}{{\mathcal L}}
\newcommand{\StError}{{s.e.}}
\newcommand{\ConfInterval}{{\mathcal I}}

%
%   Вектора
%
\newcommand{\vconst}{{\mathbf const}}
\newcommand{\vectx}{{\bm x}}
\newcommand{\vecty}{{\bm y}}
\newcommand{\vectz}{{\bm z}}
\newcommand{\vecte}{{\bm e}}
\newcommand{\vectw}{{\bm w}}
\newcommand{\vecth}{{\bm h}}
\newcommand{\vectr}{{\bm r}}
\newcommand{\vectq}{{\bm q}}
\newcommand{\vectf}{{\bm f}}%{\boldsymbol{f}}
\newcommand{\vectu}{{\bm u}}
\newcommand{\vectv}{{\bm v}}
\newcommand{\vectalpha}{{\bm{\alpha}}}
\newcommand{\vectbeta}{{\bm{\beta}}}
\newcommand{\vectgamma}{{\bm{\gamma}}}
\newcommand{\vectdelta}{{\bm{\delta}}}
\newcommand{\vectomega}{{\bm{\omega}}}
\newcommand{\vecttheta}{{\bm{\theta}}}
\newcommand{\vecteta}{{\bm{\eta}}}
\newcommand{\vectpi}{{\bm{\pi}}}
\newcommand{\vectmu}{{\bm{\mu}}}
\newcommand{\vectxi}{{\bm{\xi}}}
\newcommand{\vectX}{{\bm X}}
\newcommand{\vectY}{{\bm Y}}
\newcommand{\vectZ}{{\bm Z}}
\newcommand{\vectones}{{ 1}}

% 
%  Матрицы
%
\newcommand{\Id}{I}
\newcommand{\matrixX}{{\bm X}}
\newcommand{\matrixY}{{\bm Y}}
\newcommand{\matrixU}{{\bm U}}
\newcommand{\matrixV}{{\bm V}}
\newcommand{\matrixR}{{\bm R}}
\newcommand{\matrixZ}{{\bm Z}}
\newcommand{\matrixA}{{\bm A}}
\newcommand{\matrixB}{{\bm B}}
\newcommand{\matrixQ}{{\bm Q}}
\newcommand{\matrixH}{{\bm H}}
\newcommand{\matrixXX}{X}
\newcommand{\matrixGamma}{{\bm{\Gamma}}}
\newcommand{\matrixPi}{{\bm{\Pi}}}

%
% Теоремы, Примеры etc
%
\newtheorem*{teorema}{Теорема}
\newtheorem*{importante}{Важно!}
\newtheorem*{ejemplo}{Пример}
\newtheorem*{definicion}{Определение}

\theoremstyle{remark}
\newtheorem*{remark}{Замечание}


% \theoremstyle{plain}
% \newtheorem*{trm}{Теорема}
% \newtheorem*{mprtnt}{Важно!}
% \newtheorem*{xmpl}{Пример}
% \newtheorem*{dfntn}{Определение}

% \theoremstyle{remark}
% \newtheorem*{rmrk}{Замечание}

\title{Листок 01. Введение в Python}

\author{Н.В. Артамонов}

\begin{document}

\maketitle

\tableofcontents

\section{Pandas}

\begin{exercise}
Загрузите датасет \texttt{sleep75}.
\begin{enumerate}
	\item вычислите размер датасета (число наблюдений \& число переменных)
	\item Заполните следующую таблицу со значениями переменных
	\begin{center}
		\begin{tabular}{|c|c|c|c|c|} \hline
			index & sleep & totwrk & age & male\\ \hline\hline
			0 & & & & \\ \hline
			5 & & & & \\ \hline
			100 & & & & \\ \hline
			700 & & & & \\ \hline
		\end{tabular}
	\end{center}
	\item Вычислите корреляционную матрицу для следующих переменных: sleep, totwrk, age 
	\item Заполните следующую таблицу
	\begin{center}
		\begin{tabular}{|c|c|c|c|c|} \hline
			Desc.Stat & sleep & totwrk & age & hrwage\\ \hline\hline
			max & & & & \\ \hline
			min & & & & \\ \hline
			mean & & & & \\ \hline
			median & & & & \\ \hline
			st.dev & & & & \\ \hline
			var (unbiased) & & & & \\ \hline
			var (biased) & & & & \\ \hline
			1st quartile & & & & \\ \hline
			3rd quartile & & & & \\ \hline
		\end{tabular}
	\end{center}
	Замечание: 1st/3rd квантили -- 25\%/75\% квантили соответственно.
	\item Сколько наблюдения соответствуют следующим условиям
	\begin{enumerate}
		\item sleep>3000
		\item totwrk<2000
		\item age>40
		\item age<30
	\end{enumerate}
	\item Сколько наблюдений с условием totwrk=0? 
	Кто эти люди?
	\item Есть ли в датасете пропущенные наблюдения?
	Сколько их?
\end{enumerate}
\end{exercise}

\begin{exercise}
Загрузите датасет \texttt{Electricity}.
\begin{enumerate}
	\item вычислите размер датасета (число наблюдений \& число переменных)
	\item заполните следующую таблицу со значениями переменных
	\begin{center}
		\begin{tabular}{|c|c|c|c|c|c|} \hline
			index & cost & q & pl & pk & pf \\ \hline\hline
			1 & & & & & \\ \hline
			15 & & & &  & \\ \hline
			48 & & & & & \\ \hline
			87 & & & & & \\ \hline
		\end{tabular}
	\end{center}
	\item Вычислите корреляционную матрицу для следующих переменных: cost, q, pl, pk, pf 
	\item Заполните следующую таблицу
	\begin{center}
		\begin{tabular}{|c|c|c|c|c|c|} \hline
			Desc.Stat & cost & q & pl & pk & pf\\ \hline\hline
			max & & & & & \\ \hline
			min & & & & & \\ \hline
			mean & & & &  & \\ \hline
			median & & & & & \\ \hline
			st.dev & & & & & \\ \hline
			var (unbiased) & & & & & \\ \hline
			var (biased) & & & & & \\ \hline
			1st quartile & & & & & \\ \hline
			3rd quartile & & & & & \\ \hline
		\end{tabular}
	\end{center}
	Замечание: 1st/3rd квантили -- 25\%/75\% квантили соответственно.
	\item Сколько наблюдения соответствуют следующим условиям
		\begin{enumerate}
			\item cost>40
			\item q<5000
			\item q>4000
			\item 20<cost<50
		\end{enumerate}
	\item Есть ли в датасете пропущенные наблюдения?
	Сколько их?
\end{enumerate}
\end{exercise}

\begin{exercise}
Загрузите датасет \texttt{wage2}.
\begin{enumerate}
	\item вычислите размер датасета (число наблюдений \& число переменных)
	\item заполните следующую таблицу со значениями переменных
	\begin{center}
		\begin{tabular}{|c|c|c|c|c|c|c|} \hline
			index & wage & hours& IQ & educ & exper & age \\ \hline\hline
			1 & & & & & & \\ \hline
			25 & & & &  & & \\ \hline
			179 & & & & & & \\ \hline
			800 & & & & & & \\ \hline
		\end{tabular}
	\end{center}
	\item Вычислите корреляционную матрицу для следующих переменных: wage, hours, IQ, educ, exper 
	\item Заполните следующую таблицу
	\begin{center}
		\begin{tabular}{|c|c|c|c|c|c|c|} \hline
			Desc.Stat & wage & hours& IQ & educ & exper & wage \\ \hline\hline
			max & & & & & & \\ \hline
			min & & & & & & \\ \hline
			mean & & & & & & \\ \hline
			median & & & & & & \\ \hline
			st.dev & & & & & & \\ \hline
			var (unbiased) & & & & & & \\ \hline
			var (biased) & & & & & & \\ \hline
			1st quartile & & & & & & \\ \hline
			3rd quartile & & & & & & \\ \hline
		\end{tabular}
	\end{center}
	Замечание: 1st/3rd квантили -- 25\%/75\% квантили соответственно.
	\item Сколько наблюдения соответствуют следующим условиям
		\begin{enumerate}
			\item wage>1000
			\item age<40
			\item exper>10
			\item 100<IQ<130
		\end{enumerate}
	\item Есть ли в датасете пропущенные наблюдения?
	Сколько их?
\end{enumerate}
\end{exercise}

\begin{exercise}
Загрузите датасет \texttt{Labour}. Создайте новый датасет, 
содержащий log-переменные из исходного датасета.
\end{exercise}

\begin{exercise}
Загрузите датасет \texttt{Electricity}. Создайте новый датасет, 
содержащий log-переменные из исходного датасета.
\end{exercise}

\section{Визуализация}

\begin{exercise}
Загрузите датасет \texttt{sleep75}.
\begin{enumerate}
	\item нарисуйте гистограммы для переменных sleep, totwrk, age, hrwage, educ
	\item нарисуйте гистограмму с накопление для sleep относительно male 
	\item нарисуйте гистограмму с накопление для totwrk относительно south 
	\item нарисуйте гистограмму с накопление для totwrk относительно smsa 
	\item нарисуйте диаграмму рассеяния sleep vs totwrk
	\item нарисуйте диаграмму рассеяния sleep vs totwrk с группировкой по male
	\item нарисуйте диаграмму рассеяния sleep vs age
	\item нарисуйте диаграмму рассеяния sleep vs age с группировкой по south
	\item нарисуйте диаграмму рассеяния sleep vs edu
	\item нарисуйте диаграмму рассеяния sleep vs edu с группировкой по smsa
	\item визуализируйте корреляционную матриц для следующих переменных: sleep, totwrk, age
\end{enumerate}
\end{exercise}

\begin{exercise}
Загрузите датасет \texttt{Labour}.
\begin{enumerate}
	\item нарисуйте гистограммы для переменных output, capital, labour, wage
	\item нарисуйте гистограммы для log-переменных output, capital, labour, wage
	\item нарисуйте диаграммы рассеяния output vs других переменных
	\item нарисуйте диаграммы рассеяния log(output) vs log других переменных
	\item визуализируйте корреляционную матриц для всех переменных
	\item визуализируйте корреляционную матриц для log-переменных
\end{enumerate}
\end{exercise}

\begin{exercise}
Загрузите датасет \texttt{Electricity}.
\begin{enumerate}
	\item нарисуйте гистограммы для переменных cost, q, pf, pk, pl
	\item нарисуйте гистограммы для log-переменных cost, q, pf, pk, pl
	\item нарисуйте диаграммы рассеяния cost vs других переменных
	\item нарисуйте диаграммы рассеяния log(cost) vs log других переменных
	\item визуализируйте корреляционную матриц для всех переменных
	\item визуализируйте корреляционную матриц для log-переменных
\end{enumerate}
\end{exercise}

\begin{exercise}
Загрузите датасет \texttt{diamonds}.
\begin{enumerate}
	\item нарисуйте гистограммы для переменных price, carat
	\item нарисуйте гистограммы для log-переменных price, carat
	\item нарисуйте гистограмму с накопление для price относительно cut
	\item нарисуйте гистограмму с накопление для carat относительно clarity 
	\item нарисуйте гистограмму с накопление для log(price) относительно color
	\item нарисуйте гистограмму с накопление для log(carat) относительно color
	\item нарисуйте диаграмму рассеяния price vs carat
	\item нарисуйте диаграмму рассеяния log-price vs log-carat
	\item нарисуйте диаграмму рассеяния log-price vs log-carat с группировкой по cut
	\item нарисуйте диаграмму рассеяния log-price vs log-carat с группировкой по color
	\item нарисуйте диаграмму рассеяния log-price vs log-carat с группировкой по clarity
\end{enumerate}
\end{exercise}

\begin{exercise}
Загрузите датасет \texttt{Diamond}.
\begin{enumerate}
	\item нарисуйте гистограммы для переменных price, carat
	\item нарисуйте гистограммы для log-переменных price, carat
	\item нарисуйте гистограмму с накопление для price относительно certification
	\item нарисуйте гистограмму с накопление для carat относительно clarity 
	\item нарисуйте гистограмму с накопление для log(price) относительно colour
	\item нарисуйте гистограмму с накопление для log(carat) относительно colour
	\item нарисуйте диаграмму рассеяния price vs carat
	\item нарисуйте диаграмму рассеяния log-price vs log-carat
	\item нарисуйте диаграмму рассеяния log-price vs log-carat с группировкой по certification
	\item нарисуйте диаграмму рассеяния log-price vs log-carat с группировкой по colour
	\item нарисуйте диаграмму рассеяния log-price vs log-carat с группировкой по clarity
\end{enumerate}
\end{exercise}

\end{document}